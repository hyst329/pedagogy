\documentclass[oneside]{scrartcl}
\usepackage[14pt]{extsizes}
\usepackage[left=30mm,right=10mm,top=20mm,bottom=20mm]{geometry}
\usepackage[utf8]{inputenc}
\usepackage{stix}
\usepackage{tempora}
\usepackage[backend=biber, bibstyle=gost-numeric, date=short, sorting=ntvy, sortcites=true]{biblatex}
\usepackage[russian]{babel}
\usepackage{enumerate}
\bibliography{pedagogy}
% DIRTY HACK
\renewcommand{\sffamily}{\rmfamily}
\renewcommand{\baselinestretch}{1.5}
\begin{document}
\sloppy	
	
\section*{Введение}
Высшее образование --- уровень профессионального образования, стоящий после начального и среднего. В рамках высшего образования
обучающиеся получают систематизированные знания и практические навыки, которые позволяют решать теоретические и практические задачи
по профессиональному профилю, используя и творчески развивая современные достижения науки, техники и культуры. В отличие, например, от среднего профессионального, высшее образование подразумевает проведение обучающимися довольно значительного объема 
учебно-исследовательской и научно-исследовательской деятельности.

Учебно-исследовательская и научно-исследовательская деятельность --- смежные области исследовательской деятельности обучающегося в вузе, которые нельзя рассматривать
по отдельности, поэтому здесь акцент был сделан на обоих областях.
 
В настоящее время, высшее образование становится де-факто обязательным, поэтому все большую роль играет проблема формирования
у обучающихся (студентов) готовности к научно-исследовательской работе.

В данной работе были проанализированы статьи \cite{egorova, bystrenina, prohorova, kalganova, koldina, dolganov}, авторами которых
являются сотрудники различных вузов России и СНГ, специализирующиеся на педагогике и близких к ней областях.

\section*{Статьи}
Статья Т. П. Егоровой \cite{egorova}, сотрудника кафедры прикладной математики Кубанского государственного технологического
университета, посвящена педагогическим условиям формирования готовности студентов к учебно-исследовательской деятельности
в условиях исследовательской интернет-среды. В статье имитировались традиционные формы включения студентов в исследовательские
проекты, проведено имитационное моделирование, для реализации которого был специально создан Интернет-портал 
«Студент-исследователь».

Для доказательства эффективности такой формы организации учебно-исследовательской деятельности студентов младших курсов, было
организовано педагогическое исследование, в ходе которого была подтверждена гипотеза о том, что использование исследовательского
Интернет–портала способствует формированию готовности студентов к учебно-исследовательской деятельности.

АВтор выделяет следующие педагогические условия формирования готовности в учебно-исследовательской деятельности (рассматривая студентов и преподавателей экономического
направления, однако результаты исследования в равной мере применимы и к другим научным областям):
\begin{enumerate}[1)]
	\item формирование виртуальных учебно-исследовательских и научно-ииследовательских лабораторий;
	\item привлечение к организации учебно-исследовательской деятельности студентов-экономистов преподавателей и специалистов, владеющих как навыками научно-исследовательской 
	деятельности в области экономики, так и современными интернет-технологиями;
	\item вовлечение студентов в интернет-общение в рамках специально созданной Интернет-среды с другими студентами, преподавателями и специалистами в сфере экономики, занимающимися исследованиями, с использованием различных технических средств.
\end{enumerate}

В ходе проведения эксперимента, автору статьи удалось подтвердить гипотезу о том, что использование образовательно-исследовательского Интернет-портала позволяет значимо
улучшить уровень готовности студентов-экономистов к учебно-исследовательской деятельности и способствует повышению качества научно-исследовательских проектов, курсовых 
и выпускных квалификационных работ (к сожалению, конкретных данных, подтверждающих эту значимость, в статье не приводится).

\printbibliography
\end{document}