\documentclass[oneside]{scrartcl}
\usepackage[14pt]{extsizes}
\usepackage[left=30mm,right=10mm,top=20mm,bottom=20mm]{geometry}
\usepackage[utf8]{inputenc}
\usepackage{stix}
\usepackage{tempora}
\usepackage[backend=biber, bibstyle=gost-numeric, date=short, sorting=ntvy, sortcites=true]{biblatex}
\usepackage[russian]{babel}
\usepackage{enumerate}
\bibliography{pedagogy}
% DIRTY HACK
\renewcommand{\sffamily}{\rmfamily}
\renewcommand{\baselinestretch}{1.5}
\begin{document}
\sloppy	
	
\section*{Введение}
Высшее образование --- уровень профессионального образования, стоящий после начального и среднего. В рамках высшего образования
обучающиеся получают систематизированные знания и практические навыки, которые позволяют решать теоретические и практические задачи
по профессиональному профилю, используя и творчески развивая современные достижения науки, техники и культуры. В отличие, например, от среднего профессионального, высшее образование подразумевает проведение обучающимися довольно значительного объема 
учебно-исследовательской и научно-исследовательской деятельности.

Учебно-исследовательская и научно-исследовательская деятельность --- смежные области исследовательской деятельности обучающегося в вузе, которые нельзя рассматривать
по отдельности, поэтому здесь акцент был сделан на обоих областях.
 
В настоящее время, высшее образование становится де-факто обязательным, поэтому все большую роль играет проблема формирования
у обучающихся (студентов) готовности к научно-исследовательской работе.

В данной работе были проанализированы статьи \cite{egorova, bystrenina, prohorova, kalganova, koldina, dolganov}, авторами которых
являются сотрудники различных вузов России и Украины, специализирующиеся на педагогике и близких к ней областях.

\section*{Статьи}
\paragraph{1}
Статья Т. П. Егоровой \cite{egorova}, сотрудника кафедры прикладной математики Кубанского государственного технологического
университета, посвящена педагогическим условиям формирования готовности студентов к учебно-исследовательской деятельности
в условиях исследовательской интернет-среды. В статье имитировались традиционные формы включения студентов в исследовательские
проекты, проведено имитационное моделирование, для реализации которого был специально создан Интернет-портал 
«Студент-исследователь».

Для доказательства эффективности такой формы организации учебно-исследовательской деятельности студентов младших курсов, было
организовано педагогическое исследование, в ходе которого была подтверждена гипотеза о том, что использование исследовательского
Интернет–портала способствует формированию готовности студентов к учебно-исследовательской деятельности.

Интернет-портал «Студент-исследователь» призван помочь студентам-экономистам младших курсов включиться в исследовательскую
деятельность. На портале размещены методические материалы по основам исследовательской деятельности, лабораторные практикумы,
примеры лабораторных, исследовательских работ и форум для обсуждения, ссылки на библиотеки и электронные ресурсы, архивы
конференций и фотоархивы, информационные письма о проводимых конференциях.

Автор выделяет следующие педагогические условия формирования готовности в учебно-исследовательской деятельности (рассматривая студентов и преподавателей экономического
направления, однако результаты исследования в равной мере применимы и к другим научным областям):
\begin{enumerate}[1)]
	\item формирование виртуальных учебно-исследовательских и научно-исследовательских лабораторий;
	\item привлечение к организации учебно-исследовательской деятельности студентов-экономистов преподавателей и специалистов, владеющих как навыками научно-исследовательской 
	деятельности в области экономики, так и современными интернет-технологиями;
	\item вовлечение студентов в интернет-общение в рамках специально созданной Интернет-среды с другими студентами, преподавателями и специалистами в сфере экономики, занимающимися исследованиями, с использованием различных технических средств.
\end{enumerate}

В ходе проведения эксперимента, автору статьи удалось подтвердить гипотезу о том, что использование образовательно-исследовательского Интернет-портала позволяет значимо
улучшить уровень готовности студентов-экономистов к учебно-исследовательской деятельности и способствует повышению качества научно-исследовательских проектов, курсовых 
и выпускных квалификационных работ (к сожалению, конкретных данных, подтверждающих эту значимость, в статье не приводится).

\paragraph{2}
В статье \cite{bystrenina} преподавателя Марийского государственного университета И. Е. Быстрениной, рассматривается формирование
такой готовности в условиях интеграции математики и информатики. Автор рассматривает в первую очередь студентов педагогических
специальностей, в связи с тем, что проблема готовности педагогов к исследовательской деятельности особенно актуальна на
современном этапе развития отечественного образования.

Исследовательская деятельность студентов рассматривается автором статьи как процесс, формирующий студента путем личной
познавательной работы, направленной на получение нового знания, решение теоретических и практических вопросов, самообразование
и самореализацию своих исследовательских способностей; готовность к исследовательской деятельности учителя определяется в статье
как интегративная целостная система знаний, умений и навыков, обусловливающая качество профессионально-исследовательской
деятельности педагога в педагогической области.

Рассматривая студентов нематематических специальностей в контексте интеграции математики и информатики, автор выделяет следующие
педагогические условия:
\begin{enumerate}[1)]
\item использование различных форм интеграции математики и информатики в инвариантной и вариативной части профессиональной
подготовки будущих учителей;

\item учет образовательных потребностей и индивидуальных особенностей студентов при решении профессиональных исследовательских
задач в практике изучения математики и информатики;

\item решение исследовательских задач, связанных с педагогической деятельностью, в процессе обучения математике и информатике
с учетом принципов интегративного подхода;

\item мониторинг формирования готовности к исследовательской деятельности студентов педагогических специальностей вуза
на всех этапах вузовской подготовки. 
\end{enumerate}

Построенная модель была апробирована в Марийском государственном университете. Изначально было проведено тестирование среди 110
студентов 4-го курса педагогических специальностей гуманитарной направленности, показавшее низкую готовность студентов 
к исследовательской деятельности (55\% знали о методике исследования, 45\% имели реальный опыт, 88\% имели слабое
представление о НИД и т. п.).
Далее, была представлена программа подготовки такого формирования, включившая в себя инвариантную (курсы <<Математика и
информатика>>, <<Использование современных ИКТ в учебном процессе>>) и вариативную (авторский спецкурс <<Математические
методы в педагогических исследованиях>>). Целью спецкурса являлось формирование готовности будущего учителя к исследовательской
деятельности на основе интегративного подхода. Задачами спецкурса являлись знакомство будущих педагогов с исследовательскими
проблемами современного образования, приобретение знаний о преимуществах и потенциале использования исследовательской
деятельности в образовании, овладение специальными умениями и навыками.

В заключении статьи автор отмечает, что задачу формирования готовности студентов к исследовательской деятельности невозможно
решить только в рамках одного предмета. Необходим комплексный подход к решению данной проблемы с позиций системного,
интегративного, деятельностного, компетентностного и технологического подходов.

\paragraph{3}
Статья \cite{prohorova} написана Е. В. Прохоровой, аспирантом (на 2013 год) Южноукраинского национального педагогического
университета им. К.Д. Ушинского (г. Одесса). В ней рассматриваются уровни компетентности (грамотности, основанной на знаниях
из личного опыта) к научно-исследовательской работе студентов-магистров.

Автор выделяет 3 компонента компетентности к научно-исследовательской работе:
\begin{enumerate}[1)]
\item мотивационно-ценностный (основные показатели --- наличие профессионально значимых качеств по проведению исследований,
интерес к научно-исследовательской деятельности, удовлетворенность собственной деятельностью);
\item аналитико-деятельностный (умение анализировать профессиональную деятельность, наличие исследовательских умений,
т. е. умение систематизировать, классифицировать и оценивать собственные возможности);
\item организационно-творческий (умение представлять результаты исследовательской деятельности, способность к исследовательскому
поиску, наличие креативности, умение выбирать диагностический инструментарий).
\end{enumerate}

В экспериментальной части исследования, так же, как и в предыдущей статье, вначале автором устанавливается текущее положение дел.
Путем опроса 100 магистрантов ЮНПУ им. Ушинского были выявлены признаки недостатка компетентности у большинства субъектов
научно-исследовательской деятельности; были определены проблемы, которые необходимо учитывать при организации универсальной 
модели поэтапного и последовательного формирования научно-исследовательских компетенций.

В ходе дифференциации полученных результатов были определены 3 уровня сформированности исследовательских умений:
достаточный, умеренный и начальный.

Следующим этапом стало собственно тестирование компетентности магистров по уровням. В данном этапе приняла участие другая группа
из 100 магистров, в том числе 75 --- из ЮНПУ им. Ушинского, 13 --- Измаильского государственного гуманитарного университета,
12 --- Херсонского государственного университета.

Результаты анкетирования показали, что подавляющее большинство (72\%) магистров имеют лишь начальный уровень компетентности.
20\% обладают средним уровнем, а вот достаточный уровень показали лишь 8\% обучающихся. Это позволяет говорить о
необходимости более целенаправленной подготовки магистров к научно-исследовательской деятельности.

В связи со всем вышесказанным, автор отмечает, что анализ полученных анкетных данных позволил осознать растущую потребность
магистрантов педагогических вузов в исследовательских знаниях и умениях.

\paragraph{4}
Статья \cite{kalganova} посвящена формированию исследовательских компетенций у студентов-экономистов средствами
иностранного языка. Ее авторами являются преподаватели КФУ Г. Ф, Калганова и М. Г. Кудрявцева.

В своей статье авторы определяют задачи организации, виды научно-исследовательской работы студентов, педагогические условия
формирования исследовательских компетенций; также дается оценка уровня сформированности исследовательских умений по выделенным
компетенциям в рамках НИРС.

Участие в научно-исследовательской деятельности на кафедре иностранных языков в сфере экономики, бизнеса
и финансов Института управления, экономики и финансов КФУ разделяется на 5 этапов:
\begin{enumerate}[1)]
\item выбор темы в рамках заявленных секций, составление плана доклада и обсуждение с научным руководителем направлений
исследования, формулирование целей, гипотезы и методов изучения проблемы;
\item исследование источников, данных отечественной и зарубежной статистики на английском и русском языках;
\item проведение собственного эксперимента или опроса, составление графиков;
\item оформление презентации на английском языке с использованием специализированного программного обеспечения;
\item публичные доклады по результатам своего научного исследования и участие в дискуссиях, как с экспертом, так и
с другими участниками конференции на иностранном языке.
\end{enumerate}

Авторы статьи провели опрос, участниками которого стал 21 студент ИУЭФ КФУ (на 2017 год --- 1-го и 2-го курсов).
Опрос затрагивал вопросы мотивации и формирования научно-исследовательских умений. В ходе данного опроса авторами были 
выделены следующие 5 компонентов компетентности, соответствующие этапам:
\begin{enumerate}[1)]
\item умение работать с зарубежными источниками, анализировать данные отечественной и зарубежной статистики
 по теме исследования;
\item овладение новейшими теориями, методами и технологиями;
\item умение интерпретировать и обобщать результаты исследования в виде письменного доклада на иностранном языке; 
\item умение оформлять презентации на английском языке; 
\item умение публично докладывать результаты научного исследования и участвовать в дискуссиях по теме исследования.
\end{enumerate}

В результатах исследования авторы отмечают, что большинство студентов, принявших участие в опросе, продемонстрировали достаточно
высокий уровень сформированности научно-исследовательской компетентности, что свидетельствует об эффективной системе
организации научно-исследовательской работы студентов.

\paragraph{5}
Статья \cite{koldina}, опубликованная доцентом Нижегородского государственного педагогического университета им. Козьмы Минина
М. И. Колдиной, посвящена формированию готовности к научно-исследовательской деятельности будущих бакалавров профессионального
обучения.

В данной статье рассматриваются структура и содержание понятия <<готовность студентов к научно-исследовательской деятельности>>,
дается характеристика его основных структурных компонентов. Рассматривается проблема формирования готовности будущих бакалавров
профессионального обучения к научно-исследовательской деятельности через этапы формирования научно-исследовательских умений.

Автор уделяет много внимания понятиям <<компетентность>> и <<готовность>> (применительно к научно-исследовательской деятельности),
выясняет сходство и различия между ними. Она делает вывод, что <<компетентность>> и <<готовность>> --- ключевые понятия
профессиональной подготовки современного специалиста, находящиеся в тесной взаимосвязи друг с другом. 
Компетентность предполагает наличие готовности к определенному виду деятельности, однако говорить о замещении
категории <<готовность>> понятием <<компетентность>> не представляется возможным.

В соответствии с вышесказанным и на основании анализа научной литературы и с опорой на сущность понятия 
<<научно-исследовательская деятельность студентов>> автор самостоятельно формулирует определение готовности студентов к
научно-исследовательской деятельности, которая рассматривается как личностное образование, обусловливающее состояние личности
субъекта и включающее мотивационно-ценностное отношение к этой деятельности, систему методологических знаний, систему
исследовательских умений, позволяющих продуктивно их использовать при решении профессионально-педагогических задач.

В исследовании автором были разработаны следующие педагогические условия реализации процесса формирования готовности студентов к
научно-исследовательской деятельности:

\begin{enumerate}[1)]
\item  ориентация будущего бакалавра профессионального обучения на формирование научно-исследовательских умений;
\item определение научно-исследовательских умений (аналитико-исследовательские, модельно-прогностические,
организационно-методические, профессионально-поисковые, рефлексивно-оценочные);
\item обучение через научно-исследовательскую деятельность (разработка творческих заданий, исследовательских ситуаций,
участие студентов в научно-исследовательских конференциях);
\item определение направлений научно-исследовательской деятельности;
\item внедрение педагогических инноваций в образовательный процесс;
\item разработка и реализация учебного курса «Научно-исследовательская работа: методология, теория, практика организации и
проведения», обеспечивающего готовность будущих бакалавров профессионального обучения к научно-исследовательской деятельности.
\end{enumerate}

Аналогично \cite{prohorova}, автор также выделяет 3 уровня сформированности научно-исследовательских умений: высокий, средний
и низкий.

Проведенное экспериментальное исследование также было полностью аналогично \cite{prohorova} (см. выше). В нем приняли
участие 300 студентов Нижегородского инженерно-педагогического университета (100 человек --- экспериментальная группа, 
200 --- контрольная).

Среди опрошенных также преобладал низкий уровень сформированности (35--60\% в экспериментальной группе и 37--60\% в контрольной);
средний уровень показали 37--60\% и 37--58\%, высокий --- 3--8\% и 2--6\% соответственно. После проведения курса результаты
экспериментальной группы существенно улучшились --- до 10--18\% высокого уровня, 52--71\% среднего и лишь 18--40\% низкого,
в контрольной же остались практически без изменений (2--6\%, 38--60\% и 35--59\% соответственно), что свидетельствует
об эффективности применяемой методики (можно утверждать это с вероятностью более 98\%).

\paragraph{6}
Статья \cite{dolganov}, авторы которой --- Д. Н. Долганов, Л. И. Законнова и М. Е. Седовских --- представляют Кузбасский
государственный технический университет имени Т. Ф. Горбачева, описывает мотивационную готовность и отношение студентов
технического вуза к осуществлению научно-исследовательской деятельности.

Целью работы являлся анализ мотивации студентов к осуществлению научно-исследовательской деятельности, а также анализ
диагностических возможностей психографического теста В.Г. Леонтьева для диагностики мотивации научно-исследовательской
деятельности.

Авторы применяли следующие методы исследования:
\begin{enumerate}[1)]
\item методика диагностики мотивации научно-исследовательской деятельности Т.В. Огородовой; 
\item методика изучения мотивационного профиля Ш. Ричи и П. Мартин;
\item методика диагностики структуры мотивации труда К. Замфир;
\item психографический тест В.Г. Леонтьева.
\end{enumerate}
Статистическая обработка данных исследования проводилась с использование MS Excel и Statistica v10.

В исследовании приняли участие студенты филиала. 1-я группа --- студенты специальности <<Государственное и муниципальное
управление>>, 11 человек, на момент исследования учащиеся 4 курса. 2-я группа --- студенты специальности <<Горное дело>>, 
8 человек, на момент исследования учащиеся 4 курса. 3-я группа --- студенты специальности <<Горное дело>>, 9 человек, 
на момент исследования учащиеся 1 курса.

По утверждению авторов, полученные результаты позволяют охарактеризовать отличия в отношении к научно-исследовательской
деятельности и готовности к осуществлению научно-исследовательской деятельности студентов различных направлений. Результаты
корреляционного и факторного анализа свидетельствуют о высоком диагностическом потенциале психографического теста.

\section*{Заключение}

TODO


\printbibliography
\end{document}